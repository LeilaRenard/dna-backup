\documentclass[a4paper]{report}

% Set page dimentions
\usepackage[margin=35mm]{geometry}

% Packages for french documents
\usepackage[french]{babel}  % latex rules for french words
\usepackage[utf8]{inputenc} % UTF-8 encoding for special chars
\usepackage[T1]{fontenc}    % T1 font for smooth render of special chars
\DeclareUnicodeCharacter{202F}{\thinspace}

% Define some colors
\usepackage{color}
\definecolor{string}{RGB}{100, 200, 0}
\definecolor{comment}{RGB}{150, 150, 150}
\definecolor{identifier}{RGB}{100, 100, 200}

% Source code style
\usepackage{listings}
\lstset{
	basicstyle=\footnotesize\ttfamily, % sets font style for the code
	frame=single,                 % adds a frame around the code
	showstringspaces=false,       % underline spaces within strings
	tabsize=4,                    % sets default tabsize to 2 spaces
	breaklines=true,              % sets automatic line breaking
	breakatwhitespace=true,       % sets if automatic breaks should only happen at whitespace
	keywordstyle=\color{magenta}, % sets color for keywords
	stringstyle=\color{string},   % sets color for strings
	commentstyle=\color{comment}, % sets color for comments
	emphstyle=\color{identifier}, % sets color for comments
}

% Hyperlinks
\usepackage[hyphens]{url}
\usepackage[hidelinks]{hyperref}

% Bibliography
\usepackage[nottoc]{tocbibind}  % add link to table of contents

% Appendices
\usepackage{appendix}
\renewcommand{\appendixname}{Annexe}
\renewcommand{\appendixtocname}{Annexes}

% Graphics
\usepackage{graphicx}        % images and figures
\graphicspath{ {assets} }    % path containing images
\usepackage{tikz}            % to generate graphics
\usetikzlibrary{arrows.meta} % setup arrows
\usetikzlibrary{chains,decorations.pathreplacing} % tiks chains
\tikzstyle{Arrow}=[-{Stealth[scale=1.5]}]

% Complex tables
\usepackage{multirow,tabularx}
% Custom column types
\usepackage{array}
\newcolumntype{L}{>{\raggedright\arraybackslash}X} % Left-aligned auto-span columns
\newcolumntype{R}{>{\raggedleft\arraybackslash}X} % Right-aligned auto-span columns

% Directory tree
\usepackage{dirtree}

% Acronyms
\usepackage[printonlyused,footnote]{acronym}

%--------------------------------------- Content ---------------------------------------%

\title{Système de fichiers pour le stockage d’informations numériques sur ADN}

\date{Octobre 2021}

\author{Nicolas Peugnet}

\begin{document}

\maketitle

\tableofcontents

\chapter{Introduction}

Ce stage a été réalisé dans le cadre du projet DNA-Drive, un système développé par l'équipe de Stéphane Lemaire (IBPC) visant à stocker des données numériques arbitraires via des molécules d'\ac{adn}.

\section{Stocker des données sur ADN}

L’\ac{adn} ou acide désoxyribonucléique d’un organisme, contient ce qu’on appelle le génome.
Le génome est l’information génétique d’un organisme. L’\ac{adn} contient donc une information.
Cette information est codée sous la forme d’une suite de \emph{nucléotides}.
Un nucléotide est une molécule organique qui est l’élément de base de l’\ac{adn}.
Il existe 4 types de nucléotides différents qui sont représentés par 4 lettres : \textbf{A} pour Adénine, \textbf{C} pour Cytosine, \textbf{G} pour Guanine et \textbf{T} pour Thymine.
Nous pouvons voir directement le parallèle que nous pouvons faire entre l’\ac{adn} qui est une suite de nucléotides en base 4 et une donnée informatique qui est une suite de bits en base 2.

Il est donc naturel de penser à utiliser l’\ac{adn} pour stocker des données
et un certain nombre de démonstrations de faisabilité du stockage sur l’\ac{adn} ont été réalisées lors des dernières années.
Les travaux publiés pour l’instant se basent essentiellement sur l’utilisation d’\emph{oligonucléotides} qui sont des courts segments d’\ac{adn}.


\subsection{Spécificités des supports de stockage ADN}

Différents encodages et techniques de conservation sont utilisés, mais elles ont en commun quelques spécificités :

\begin{enumerate}
  \item Elles ne permettent pas de supprimer des données une fois écrites.
  \item Les lectures sont lentes et coûteuses.
  \item Les écritures le sont encore plus.
\end{enumerate}

On considère que l'encodage utilisé pour stocker les données sur \ac{adn} nous permet de ne pas nous soucier des contraintes biologiques sous-jacentes, et nous permet donc d'écrire des données totalement arbitraires, comme c'est le cas de l'encodage BIODATA utilisé dans le cadre de ce projet.

\subsection{Spécificités du DNA-Drive}

La spécificité principale de DNA-Drive par rapport à ses concurrents est d'utiliser la molécule d'\ac{adn} sous sa forme de double hélice, plutôt que sous la forme d'un simple brin.
Cette forme a plusieurs avantages.
Premièrement, la molécule est plus stable sous cette forme ce qui permet de limiter sa dégradation et donc d'augmenter sa durée de vie.
Deuxièmement, cette forme est celle utilisé par l'ensemble des organismes vivants de notre planète\footnote{En considérant que les virus ne sont pas vivants},
ce qui nous permet donc potentiellement de profiter des mécanismes du vivant,
tels que la réparation automatique de l’\ac{adn} pour corriger les erreurs
ou la mitose qui va permettre une copie peu coûteuse et très rapide de grande quantité de données.

\section{Problématique}

Une des fonctionnalités que le système devait supporter était la possibilité de mettre à jour des fichiers déjà écrits.
Or le médium de stockage utilisé ne permet ni de supprimer des données écrites, ni même de les modifier.
Cette problématique se retrouve sur d'autres médiums de stockages, comme les bandes magnétiques ou les disques optiques.
Elle est résolue par leur système de fichiers respectif, \ac{ltfs} \cite{pease2010linear} pour les bandes magnétiques et \ac{udf} \cite{optical2003universal} pour les CD et DVD non-RW.
Dans les deux cas le système est basé sur la réécriture complète des blocs modifiés des fichiers ainsi que de l'index dans le cas de \ac{ltfs} ou de la Virtual Allocation Table dans le cas de \ac{udf}.

% ajouter un § par rapport aux coùt des écritures

La difficulté principale était donc de réussir à implémenter cette fonctionnalité sur un médium de stockage qui n'a pas la capacité de modifier les données existantes, tout en limitant les écritures au strict nécessaire.


\section{Réponse}

La proposition qui suit s'inscrit dans le cadre d'une réponse à court terme au problème posé.
Nous avons choisi de na pas nous projeter trop loin dans le temps et avons donc basé l'ensemble de la réflexion sur les capacités actuelles des technologies de synthèse et de séquençage \ac{adn}.
L'objectif principal du système d'archivage de fichiers proposé est de réduire la quantité de données écrites, tout minimisant la quantité de données à lire pour récupérer les données.

Toutes les contraintes citées précédemment nous ont incité % j'aime bof ce mot
à nous orienter plus vers un système de sauvegardes que vers un véritable système de fichier.
En effet, les vitesses et coûts d'écriture et de lecture ne permettent, pour le moment, absolument pas d'en faire un système de fichier accessible à chaud.
Les cas d'usage envisagés seront donc ceux de sauvegardes sur différentes plages de temps : journalières, hebdomadaires ou mensuelles.
De cette manière, l'ensemble des opérations réalisés sur les fichiers pendant cette plage de temps seront factorisées dans un seul bloc de modification : la nouvelle version.
Ce n'est effectivement pas la peine d'écrire un fichier s'il va être supprimé ou renommé quelques secondes plus tard.

Afin de minimiser la quantité de données écrites par version, celles-ci sont réalisées de manière incrémentale.
Chaque nouvelle version est donc en quelque sorte une différence par rapport aux précédentes.
Ce stockage incrémental est obtenu grâce à une utilisation conjointe de la déduplication et de l'encodage delta.
De plus, comme aucune donnée ne peut être supprimée, nous en profitons pour réaliser un système versionné, qui nous laisse la possibilité d'accéder aux précédentes sauvegardes.




La solution proposée se base sur une technique de  \cite{shilane2012wan}.

\chapter{Présentation générale}

DNA-Backup tien donc plus 

Le système part du principe qu'on a une copie des données stockées en
\ac{adn} sur un support de stockage classique : le \emph{repo} (Figure \ref{fig:big-picture}).
\begin{figure*}[ht]
\centering

\begin{tikzpicture}

\draw (0,0) node[anchor=south west] {Ordinateur} rectangle (8, 3.5) ;

\draw (.5,1) rectangle (3,3) node[midway] {Source};
\draw (5,1) rectangle (7.5,3) node[midway] {Repo};
\draw (10,1) rectangle (12.5,3) node[midway] {DNA-Drive};

\draw[Arrow] (3,2) -- (5,2) node[midway,below] {Commit};
\draw[Arrow] (7.5,2) -- (10,2) node[midway,below] {Synthèse};

\end{tikzpicture}

\caption{Schéma global}
\label{fig:big-picture}
\end{figure*}

La Figure \ref{fig:repo-dir-tree} montre la structure du \emph{repo}. 

\begin{figure}
% \centering % centering ne fonctionne pas du tout avec le dirtree
\dirtree{%
.1 repo/.
.2 00000/.
.3 chunks/.
.4 000000000000000.
.4 000000000000001.
.4 000000000000002.
.4 000000000000003.
.3 files.
.3 hashes.
.3 recipe.
.2 00001/.
.3 chunks/.
.4 000000000000000.
.4 000000000000001.
.3 files.
.3 hashes.
.3 recipe.
}
\caption{Organisation du \emph{repo}}
\label{fig:repo-dir-tree}
\end{figure}

Pour un repo d'une taille totale de 401 Mio :

\begin{table}[ht]
\centering
\begin{tabular}{l r r}
\verb|repo/00000/recipe| &   5076011 &   1.2\% \\
\verb|repo/00000/files| &      24664 &   0.1\% \\
\verb|repo/00000/hashes| &   3923672 &   0.9\% \\
\verb|repo/00000/chunks| & 412263137 &  97.8\% \\
\verb|repo/00000| &        421287604 & 100.0\% \\
\end{tabular}
\caption{Répartition des données d'une première version}
\label{fig:repo-data-distribution}
\end{table}

\begin{itemize}
\item
  On considère que le \emph{repo} est toujours présent lors d'une
  écriture (\emph{commit}).
\item
  Le \emph{repo} peut être reconstruit à partir des données présentes
  dans le \emph{DNA-Drive}.
\item
  Les \emph{hashes} ne sont pas écrits en \ac{adn}, car ils peuvent être
  reconstruits à partir des données des \emph{chunks}.
\item
  L'ensemble des données écrites en \ac{adn} sont compressées, pour le moment
  via \emph{ZLib}.
\item
  Les métadonnées sont stockées de manière incrémentale, chaque version
  stocke donc ses métadonnées sous la forme de delta par rapport à la
  version précédente.
\end{itemize}

On imagine le \emph{DNA-Drive} comme un segment de \emph{pools} (Figure \ref{fig:data-layout})

\begin{figure}[ht]
\centering

\begin{tikzpicture}[
  start chain = going right,
  node distance = 0,
  Box/.style={draw, minimum width=2em, minimum height=2em, outer sep=0, on chain},
  Brace/.style={decorate,decoration={brace, amplitude=1em, raise=.5em, mirror}}
]
\node[Box] (p0) {$0$};
\node[Box] (p1) {$1$};
\node[Box] (p2) {$2$};
\node[Box] (p3) {$3$};
\node[Box] (p4) {$4$};
\node[Box,minimum width=6em] (ellipsis) {$\cdots$};
\node[Box] (p93) {$93$};
\node[Box] (p94) {$94$};
\node[Box] (p95) {$95$};

\draw[Arrow] (p4.east) to +(2em,0);
\draw[Arrow] (p93.west) to +(-2em,0);

\node (ver) at (0,-3.2em) {version};
\draw[->] (p0.south) to (ver);
\draw[Brace] (p1.south west) to node[black,midway,below=1.5em] {chunks} (p4.south east);
\draw[Brace] (p93.south west) to node[black,midway,below=1.5em,align=center] {metadata\\(recipe+files)} (p95.south east);

\end{tikzpicture}

\caption{Disposition des données}
\label{fig:data-layout}
\end{figure}


\chapter{Fonctionnement détaillé}

\section{Algorithme du commit}

\begin{enumerate}
\item
  Chargement des métadonnées du \emph{repo} afin de reconstruire en
  mémoire l'état de la dernière version :

  \begin{itemize}
  \item
    Reconstruction de la \emph{recipe} à partir des deltas de chaque
    version.
  \item
    Reconstruction du listage des fichiers à partir des deltas de chaque
    version (fichier \emph{files}).
  \item
    Reconstruction en mémoire des \emph{maps} de \emph{fingerprints} et
    de \emph{sketches} à partir des fichiers \emph{hashes} de chaque
    version.
  \end{itemize}
\item
  Listage des fichiers de la \emph{source}.
\item
  Concaténation de l'ensemble des fichiers de la source en un disque
  virtuel continu.
\item
  Lecture du \emph{stream} de ce disque virtuel et découpage en
  \emph{chunk} (de 8 Kio actuellement).
\item
  Pour chaque \emph{chunk} du \emph{stream} :

  \begin{enumerate}
  \item
    Calculer sa \emph{fingerprint} (hash classique), si elle est
    présente dans la \emph{map} : le stocker de manière dé-dupliquée
    (sous la forme d'identifiant faisant référence au \emph{chunk}
    trouvé dans la map).
  \item
    Sinon, calculer son \emph{sketch} (hash de ressemblance), s'il est
    présent dans la \emph{map}, le stocker sous la forme de delta
    (calcul de sa différence par rapport au \emph{chunk} trouvé dans la
    map).
  \item
    Sinon, le stocker sous la forme de nouveau bloc (ajout de sa
    \emph{fingerprint} et de son \emph{sketch} dans les \emph{maps} et
    stockage du contenu complet dans un nouveau \emph{chunk}).
  \end{enumerate}
\item
  Calcul des différences entre la nouvelle version et la précédente pour
  les métadonnées (\emph{files} et \emph{recipe}) et stockage des deltas
  ainsi obtenus.
\end{enumerate}

\section{Algorithme du restore}

\begin{enumerate}
\item
  Chargement des métadonnées du \emph{repo} afin de reconstruire en
  mémoire l'état de la dernière version :

  \begin{itemize}
  \item
    Reconstruction de la \emph{recipe} à partir des deltas de chaque
    version.
  \item
    Reconstruction du listage des fichiers à partir des deltas de chaque
    version.
  \end{itemize}
\item
  À partir de la \emph{recipe}, reconstruire le disque virtuel (sous la
  forme d'un \emph{stream}).
\item
  Découper ce \emph{stream} en fonction du listage des fichiers
  (\emph{files}) et réécrire les données dans les fichiers
  correspondants dans le répertoire \emph{destination}.
\end{enumerate}

\section{\texorpdfstring{Restaurer sans le
\emph{repo}}{Restaurer sans le repo}}

\subsection{\texorpdfstring{Reconstruction complète du
\emph{repo}}{Reconstruction complète du repo}}

Il est possible de reconstruire le \emph{repo} en entier en lisant la
totalité du \emph{DNA-Drive}.

\subsection{Restauration de la dernière
version}

Il est possible de ne restaurer que la dernière version en lisant dans
un premier temps le \emph{pool} de versions et les quelques \emph{pools}
de métadonnées (environ 2\% de la totalité des données écrites), puis en
lisant tous les \emph{pools} contenant des \emph{chunks} référencés par
la \emph{recipe} de cette version.

\subsection{Restauration d'un seul
fichier}

Il pourrait être possible (pas pour le moment) de ne restaurer qu'un
seul fichier d'une version en ayant moins de données à lire que pour
restaurer la version complète.

Pour cela, il faudrait en plus stocker en \ac{adn} un mapping \emph{chunk}
décompressé → \emph{pool} contenant ce \emph{chunk} et ainsi n'avoir à
lire que les \emph{pools} contenant des \emph{chunks} de ce fichier.

\chapter{Évaluation de performances}

Le dossier \verb|exp| contient les scripts permettant de reproduire
les expériences. Les scripts ne sont prévus pour fonctionner que sur
Linux.

On utilise le dépôt Git du noyau Linux comme base de donnée de test. Il
s'agit en effet d'une bonne simulation de modification de dossiers, car
l'historique contient toutes les modifications qui ont été apportées
petit à petit à l'ensemble des fichiers.

\section{Bases de comparaison}

Pour évaluer les performances du système DNA-Backup, quatre autres
systèmes de stockage versionnés ont été choisis comme base de
comparaison :

\begin{itemize}
\item
  \textbf{Git diffs}
\item
  \textbf{Git objects}
\item
  \textbf{Tar.gz}
\item
  \textbf{Taille réelle}
\end{itemize}

\subsection{Git diffs}

Ce système utilise le delta généré par la commande \verb|git diff|
pour sauvegarder une nouvelle version. Les données à stocker consistent
donc en une somme de deltas. Pour restaurer les données, il faut
appliquer séquentiellement l'ensemble des deltas jusqu'à obtenir l'état
de la version voulue.

\subsection{Git objects}

Ce système nous permet de simuler un système de fichier qui ne serait
pas autorisé à modifier des données sur le support tout en gardant la
possibilité de modifier les données. Il s'agit de la manière dont Git
sauvegarde les données des fichiers d'un dépôt. Le contenu de chaque
fichier et de chaque dossier est hashé afin d'en obtenir une signature.
Il est ensuite compressé et stocké sous la forme d'\emph{object}
immuable, référencé par la signature obtenue. Si un fichier est modifié,
il produira une signature différente et sera donc stocké sous la forme
d'un nouvel \emph{object}. Par contre, si deux fichiers ont un contenu
strictement identique, ils produiront alors la même signature et seront
donc automatiquement dé-dupliqués. Les dossiers sont également stockés
en tant qu'\emph{objects}, mais les fichiers qu'ils contiennent sont
référencés non pas par leur nom, mais par leur signature. La
modification d'un fichier entrainera donc l'ajout de nouveaux
\emph{objects} pour l'ensemble des dossiers de la branche contenant ce
fichier. C'est de cette manière que Git est capable de créer un système
de fichiers modifiable à partir d'objets immuables.

\subsection{Tar.gz}

Une technique d'archivage assez classique à laquelle il peut être
intéressant de nous comparer est de stocker chaque version en tant
qu'une nouvelle archive Tar elle-même compressée à l'aide de Gzip. Cette
technique produit des archives d'une taille très réduite, car la
compression est appliquée à l'ensemble des fichiers d'un seul coup,
contrairement à une compression fichier par fichier.

Elle a cependant l'inconvénient de ne pas faire de dé-duplication ni
d'encodage delta, et ne tire donc pas du tout parti des données déjà
écrites sur le support.

\subsection{Taille réelle}

Cette base de comparaison n'est en réalité pas un système viable. Elle
correspond à la taille que prend en réalité le dossier \emph{source} au
moment de la sauvegarde. C'est un indicateur qui permet de se rendre
compte du poids que prendrait la sauvegarde de multiples versions sans
aucune déduplication ou compression.

\subsection{Tableau récapitulatif}

\begin{table*}[ht]

\begin{tabularx}{\textwidth}{L|L|L|L|L|L}

\textbf{Feature} &
\textbf{DNA-Backup} &
\textbf{Git diffs} &
\textbf{Git objects} &
\textbf{Tar.gz} &
\textbf{Taille réelle} \\
\hline

\multirow{2}{=}{Dé\-du\-pli\-ca\-tion} &
Niveau chunk &
\multirow{2}{=}{N/A} &
Niveau fichier &
\multirow{2}{=}{N/A} &
\multirow{2}{=}{N/A} \\ \cline{2-2} \cline{4-4}
& Transversal aux versions & & Transversal aux versions & \\
\hline

\multirow{2}{=}{Delta-encoding} &
Niveau chunk &
Niveau version &
\multirow{2}{=}{N/A} &
\multirow{2}{=}{N/A} &
\multirow{2}{=}{N/A} \\ \cline{2-3}
& Transversal aux versions & Par rapport à la précédente & & \\
\hline

Com\-pres\-sion &
Niveau chunk &
Niveau version &
Niveau fichier &
Niveau version &
N/A \\
\hline

Res\-tau\-ra\-tion de la dernière version &
Lecture des métadonnées puis des chunks de cette version (répartis dans différents pools) &
Lecture de la totalité du DNA-Drive &
Lecture récursive des différents objets composant la version (répartis dans différents pools) &
Lecture de la zone correspondant à la dernière version &
Lecture de la zone correspondant à la dernière version \\

\end{tabularx}

\caption{Tableau récapitulatif}
\label{tab:recap-table}
\end{table*}

\section{Nombre d'octets par version}

\subsection{Légende}

\begin{itemize}
\item
  \verb|4k_export| : le système DNA-Backup avec des blocs de 4 Kio.
\item
  \verb|8k_export| : le système DNA-Backup avec des blocs de 8 Kio.
\item
  \verb|diffs| : une somme de diffs Git minimales Gzippées.
\item
  \verb|nopack| : le dossier `objects de Git, contenant l'ensemble des
  données des fichiers et dossiers d'un dépôt.
\item
  \verb|targz| : une somme d'archives Tar Gzippées.
\item
  \verb|real| : le poids réel de chaque version et donc l'espace
  nécessaire à stocker l'ensemble des versions de manière
  non-dédupliquées.
\end{itemize}

\subsection{Résultats}


\begin{table*}[ht]
\centering
\begin{tabularx}{\textwidth}{RRRRRRRR}
\textbf{Repo 4k} &
\textbf{Repo 8k} &
\textbf{DNA 4k} &
\textbf{DNA 8k} &
\textbf{Git diffs} &
\textbf{Git objects} &
\textbf{Tar.gz} &
\textbf{Taille réelle} \\
\hline
22 377 & 19 333 & 13 260 & 12 240 & 10 147 & 14 601 & 10 739 & 147 538 \\
2 374 & 3 138 & 3 060 & 4 080 & 1 768 & 2 573 & 11 304 & 149 715 \\
2 116 & 2 747 & 3 060 & 4 080 & 867 & 2 681 & 11 611 & 150 955 \\
3 012 & 2 917 & 4 080 & 4 080 & 869 & 2 359 & 12 111 & 152 087 \\
5 210 & 5 430 & 5 100 & 6 120 & 4 373 & 6 020 & 14 192 & 165 312 \\
\hline
35 089 & 33 565 & 28 560 & 30 600 & 18 024 & 28 234 & 59 957 & 765 607 \\

\end{tabularx}
\caption{Commits journaliers}
\label{tab:commits-daily}
\end{table*}


\begin{table*}[ht]
\begin{tabularx}{\textwidth}{RRRRRRRR}
\textbf{Repo 4k} &
\textbf{Repo 8k} &
\textbf{DNA 4k} &
\textbf{DNA 8k} &
\textbf{Git diffs} &
\textbf{Git objects} &
\textbf{Tar.gz} &
\textbf{Taille réelle} \\
\hline
22 377 & 19 333 & 13 260 & 12 240 & 10 147 & 14 601 & 10 739 & 147 538 \\
2 374 & 3 138 & 3 060 & 4 080 & 1 768 & 2 573 & 11 304 & 149 715 \\
2 116 & 2 747 & 3 060 & 4 080 & 867 & 2 681 & 11 611 & 150 955 \\
3 012 & 2 917 & 4 080 & 4 080 & 869 & 2 359 & 12 111 & 152 087 \\
5 210 & 5 430 & 5 100 & 6 120 & 4 373 & 6 020 & 14 192 & 165 312 \\
\hline
35 089 & 33 565 & 28 560 & 30 600 & 18 024 & 28 234 & 59 957 & 765 607 \\

\end{tabularx}
\caption{Commits hebdomadaires}
\label{tab:commits-weekly}
\end{table*}


\begin{table*}[ht]
\begin{tabularx}{\textwidth}{RRRRRRRR}
\textbf{Repo 4k} &
\textbf{Repo 8k} &
\textbf{DNA 4k} &
\textbf{DNA 8k} &
\textbf{Git diffs} &
\textbf{Git objects} &
\textbf{Tar.gz} &
\textbf{Taille réelle} \\
\hline
22 377 & 19 333 & 13 260 & 12 240 & 10 147 & 14 601 & 10 739 & 147 538 \\
2 374 & 3 138 & 3 060 & 4 080 & 1 768 & 2 573 & 11 304 & 149 715 \\
2 116 & 2 747 & 3 060 & 4 080 & 867 & 2 681 & 11 611 & 150 955 \\
3 012 & 2 917 & 4 080 & 4 080 & 869 & 2 359 & 12 111 & 152 087 \\
5 210 & 5 430 & 5 100 & 6 120 & 4 373 & 6 020 & 14 192 & 165 312 \\
\hline
35 089 & 33 565 & 28 560 & 30 600 & 18 024 & 28 234 & 59 957 & 765 607 \\

\end{tabularx}
\caption{Commits Mensuels}
\label{tab:commits-monthly}
\end{table*}


% Bibliography
\bibliographystyle{plain}
\bibliography{doc.bib}

% Glossary
% LTeX: enabled=false

\chapter*{Glossaire}
\addcontentsline{toc}{chapter}{Glossaire}
\begin{acronym}
	\acro{cli}[CLI]{Command Line Interface}
	\acro{ltfs}[LTFS]{Linear Tape File System}
	\acro{udf}[UDF]{Universal Disk Format}
\end{acronym}


% Annexes
\appendix
\addappheadtotoc

\chapter{Documentation de la CLI}

DNA-Backup est un programme s'utilisant en \ac{cli}.
Trois commandes sont disponibles :

\begin{itemize}
  \item \verb|commit| : pour ajouter une nouvelle version au \emph{repo}.
  \item \verb|restore| : afin de restaurer la dernière version depuis le \emph{repo}
  \item \verb|export| : pour générer un export à partir des données du \emph{repo}
\end{itemize}


\end{document}
